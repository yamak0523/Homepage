\documentclass[a4j,dvipdfmx]{jarticle}
\usepackage{amsmath,amsfonts,amssymb}
\usepackage{framed}
\renewcommand{\d}{\displaystyle}
\newcommand{\ipr}[2]{\left\langle #1,#2\right\rangle}
\newcommand{\norm}[1]{\left\|#1\right\|}
\usepackage{tikz}
\usepackage{pxpgfmark}
\usepackage{geometry}
\geometry{left=25mm,right=25mm,top=15mm,bottom=15mm}
\title{実Hahn-Banachの拡張定理}
\date{}
\author{}
\pagestyle{empty}
\usepackage{here}
\usepackage{graphicx}
\begin{document}
\maketitle
\thispagestyle{empty}
\begin{description}
\begin{framed}
\item[Theorem.] $X$を実線形空間とし, $p$は$X$上で定義された実数値汎関数で
\[ p(x+y)\leq p(x)+p(y) \ (x,y\in X), p(\alpha x)=\alpha p(x) \ (\alpha\geq 0,x\in X) \]
をみたすものとする. $M$を$X$の線形部分空間とし, $f$は
\[ f(x)\leq p(x) \ (x\in M) \]
をみたす, $M$上で定義された線形汎函数とする.このとき, $X$全体で定義された線形汎関数$f_1$で
\[ f_1(x)=f(x) \ (x\in M), f_1(x)\leq p(x) \ (x\in X) \]
をみたすものが存在する.
\end{framed}

\item[Proof.] 証明が長いので, 2段階で示す.
\begin{enumerate}
\item[(1)] $L\neq X$となる$X$の線形部分空間$L$に対して, $L$上で定義された線形汎関数で$\ell(x)\leq p(x) \ (x\in L)$をみたすものとする.また, $x_0\notin L, L+[x_0]=\{x+\alpha x_0 \mid x\in L,\alpha\in\mathbb{R}\}$とする.このとき$L\subsetneq L+[x_0]$となる.このとき
\[ g(x)=\ell(x) \ (x\in L), g(x)\leq p(x) \ (x\in L+[x_0]) \]
となる線形汎関数$g$が存在することを示す. \\ \ \\
$x,y\in L$に対して
\[ \ell(x)-\ell(y)=\ell(x-y)\leq p(x-y)\leq p(x+x_0)+p(-y-x_0) \]
より$-p(-y-x_0)-\ell(y)\leq p(x+x_0)-\ell(x)$が成り立つ.ゆえに
\[ \sup_{y\in L}\left(-p(-y-x_0)-\ell(y)\right)\leq\inf_{x\in L}\left(p(x+x_0)-\ell(x)\right) \]
となるから
\[ \sup_{y\in L}\left(-p(-y-x_0)-\ell(y)\right)\leq\lambda\leq\inf_{x\in L}\left(p(x+x_0)-\ell(x)\right) \tag{☆}\]
をみたす$\lambda$が存在する.この$\lambda$を用いて$g$を
\[ g(x+\alpha x_0)=\ell(x)+\alpha\lambda \ (x+\alpha x_0\in L+[x_0])\]
と定義する.この$g$に対して, $\alpha=0$を考えることにより
\[ g(x)=\ell(x) \ (x\in L) \]
をみたす.また, $\alpha>0$のとき不等式(☆)より
\[ g(x+\alpha x_0)=\alpha\left(\ell\left(\frac{x}{\alpha}\right)+\lambda\right)\leq \alpha p\left(\frac{x}{\alpha}+x_0\right)=p(x+\alpha x_0) \]
が成り立ち, $\alpha<0$のときも同様にして$g(x+\alpha x_0)\leq p(x+\alpha x_0)$を示すことができる.よって, このように$L+[x_0]$上で定義された$g$は求める線形汎関数となる.

\item[(2)] 次に, $g$は$M\subset D(g) \subset X$なる線形部分空間$D(g)$で定義された線形汎函数で
\[ g(x)=f(x) \ (x\in M), g(x)\leq p(x) \ (x\in D(g)) \]
をみたすものとする.このような$g$の全体を$E$で表す.ここで$D(g)$は$g$の定義域である. (1)より$E\neq\varnothing$である.任意の$g,h\in E$に対して, $h$が$g$の拡張であるとき$g\prec h$と表すことにすると, $(E,\prec)$は順序集合となる. \\ \ \\
今, $F$を$E$の任意の全順序部分集合とすると, $F$は$E$の中に上界をもつことを示す. \\ \ \\
$D=\d\bigcup_{g\in F}D(g)$とすると, $D$は$M\subset D\subset X$をみたす$X$の線形部分空間である.ここで, 各$x\in D$に対して, $x\in D(g)$なる$g\in F$をとり$f_0(x)=g(x)$によって, $D$上の汎関数$f_0$を定義する.この$f_0$は一意に定まる.実際, $g,h\in F$に対して$x\in D(g)\cap D(h)$となった場合, $g\prec h$または$h\prec g$のどちらかが成り立ち,どちらの場合であっても$g(x)=h(x)$が成り立つからである.この$f_0$が線形であることおよび
\[ f_0(x)=f(x) \ (x\in M), f_0(x)\leq p(x) \ (x\in D) \]
が成り立つことは明らかである. \\ \ \\
ゆえに, $f_0\in E$であり, 任意の$g\in F$に対して$g\prec f_0$が成り立つから$f_0$は$F$の上界である.よって, Zornの補題から$E$は少なくとも1つの極大元$f_1$をもつ.この$f_1$に対して$D(f_1)=X$であることを示す. \\ \ \\
$D(f_1)\neq X$とすると, $x_1\neq D(f_1)$なる$X$の元が存在する. (1)と同様にして
\[ f_2(x)=f_1(x) \ (x\in D(f_1)), f_2(x)\leq p(x) \ (x\in D(f_1)+[x_1]) \]
をみたす線形汎関数$f_2$が構成でき,その構成法から$f_1 \precneqq f_2$となるがこれは$f_1$は$E$の極大元であることに矛盾する.ゆえに, $D(f_1)=X$であることが示された.
\end{enumerate}
以上により, 定理が示された. $\blacksquare$
\end{description}
\end{document}
